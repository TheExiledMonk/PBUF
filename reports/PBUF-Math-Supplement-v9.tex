\documentclass[12pt,a4paper]{article}
\usepackage{amsmath,amssymb,amsfonts}
\usepackage{geometry}
\usepackage{graphicx}
\usepackage{hyperref}
\usepackage{physics}
\usepackage{upgreek}
\usepackage{bm}
\geometry{margin=1in}
\hypersetup{colorlinks=true, linkcolor=blue, urlcolor=blue, citecolor=blue}

\title{\textbf{Planck-Bound Unified Framework (PBUF) — Mathematical Supplement (v9.0)}}
\author{Fabian Olesen\\Independent Researcher\\\texttt{github.com/TheExiledMonk/PBUF}}
\date{Compiled: October 20, 2025}

\begin{document}
\maketitle

\begin{abstract}
This supplement provides the formal derivation of the elastic stress tensor $\sigma_{\mu\nu}$ in the Planck-Bound Unified Framework (PBUF), proves its covariant conservation, and outlines background and perturbative stability criteria. The formulation is fully covariant, ghost-free, and consistent with gravitational-wave constraints. A perturbative appendix summarizes the main stability derivations and parameter conditions.
\end{abstract}

\section*{0. Notation and Conventions}
Signature $(-,+,+,+)$; reduced Planck mass $M_P^{-2}=8\pi G$. Einstein tensor $G_{\mu\nu}=R_{\mu\nu}-\tfrac{1}{2}g_{\mu\nu}R$.  
Matter stress-energy $T_{\mu\nu}=-\frac{2}{\sqrt{-g}}\frac{\delta(\sqrt{-g}\mathcal{L}_m)}{\delta g^{\mu\nu}}$.
The field equations take the PBUF form:
\begin{equation}
G_{\mu\nu} + \sigma_{\mu\nu} = 8\pi G\,T_{\mu\nu}.
\end{equation}

\section{Effective Action and Definition of $\sigma_{\mu\nu}$}
We start from a diffeomorphism-invariant action
\begin{equation}
S[g,\Psi]=\int d^4x\,\sqrt{-g}\left[\frac{R}{16\pi G}+\mathcal{L}_{\mathrm{elastic}}(g;\mathcal{I})+\mathcal{L}_m(g,\Psi)\right],
\end{equation}
where $\mathcal{I}$ denotes curvature invariants such as $R$, $R_{\mu\nu}R^{\mu\nu}$, etc.  
The \textbf{elastic stress tensor} is defined by
\begin{equation}
\sigma_{\mu\nu}\equiv -\frac{2}{\sqrt{-g}}\frac{\delta(\sqrt{-g}\,\mathcal{L}_{\mathrm{elastic}})}{\delta g^{\mu\nu}}.
\end{equation}
Varying the total action yields
\begin{equation}
G_{\mu\nu}+\sigma_{\mu\nu}=8\pi G\,T_{\mu\nu}.
\end{equation}

\subsection*{1.1 Bounded-$f(R)$ Realization}
Let
\begin{equation}
\mathcal{L}_{\mathrm{elastic}}=\frac{1}{16\pi G}[f(R)-R], \quad f_R\equiv\frac{df}{dR}.
\end{equation}
Then
\begin{align}
\sigma_{\mu\nu} &= (f_R-1)G_{\mu\nu} + \frac{1}{2}g_{\mu\nu}(f-Rf_R) + \nabla_{\mu}\nabla_{\nu}f_R - g_{\mu\nu}\Box f_R.
\end{align}
Boundedness of curvature is achieved with forms such as
\begin{equation}
f(R)=R_\star\tanh\!\left(\frac{R}{R_\star}\right)+\lambda R,
\end{equation}
or more generally $f(R)=R_\star^2 \beta^{-1}[1-\sqrt{1-2\beta R/R_\star^2}]$.
These ensure $\abs{R}\le R_\star$, $f_R>0$, and $f_{RR}>0$.

\section{Covariant Conservation}
Because the total action is diffeomorphism invariant,
\begin{equation}
\nabla^\mu T_{\mu\nu}=0.
\end{equation}
Taking the divergence of the field equations and using $\nabla^\mu G_{\mu\nu}=0$ yields
\begin{equation}
\nabla^\mu\sigma_{\mu\nu}=0.
\end{equation}
Hence $\sigma_{\mu\nu}$ is \textbf{covariantly conserved}, and the Bianchi identity is preserved.

\section{Background Cosmology and Effective Fluid}
On FLRW metric $ds^2=-dt^2+a(t)^2\gamma_{ij}dx^i dx^j$, we define
\[
\sigma^\mu{}_\nu = \mathrm{diag}(-8\pi G\rho_\sigma,\,8\pi G p_\sigma,\,8\pi G p_\sigma,\,8\pi G p_\sigma),
\]
leading to modified Friedmann equations:
\begin{align}
3H^2+3\frac{K}{a^2}+\sigma^0{}_0 &= 8\pi G(\rho_m+\rho_r),\\
-2\dot H - 3H^2 - \frac{K}{a^2} + \tfrac{1}{3}\sigma^i{}_i &= 8\pi G(p_m+p_r).
\end{align}
The conservation $\nabla\cdot\sigma=0$ implies
\begin{equation}
\dot\rho_\sigma + 3H(\rho_\sigma+p_\sigma)=0,
\end{equation}
so the elastic sector behaves as a self-consistent effective fluid with equation of state $w_\sigma=p_\sigma/\rho_\sigma$.

\section{Linear Perturbations and Stability}
Working in Newtonian gauge:
\begin{equation}
ds^2=-(1+2\Phi)dt^2+a^2(1-2\Psi)\delta_{ij}dx^i dx^j,
\end{equation}
we examine scalar, vector, and tensor perturbations.

\subsection*{4.1 Scalar Sector}
The quadratic action is
\begin{equation}
S^{(2)}_S=\frac{1}{2}\int d^3k\,dt\,a^3\!\left[Q_S\dot{\zeta}^2-c_S^2\frac{k^2}{a^2}\zeta^2\right],
\end{equation}
with $Q_S>0$ (no ghosts) and $c_S^2>0$ (no gradient instability).  
For bounded-$f(R)$ sectors, the conditions
\[
f_R>0,\qquad f_{RR}>0
\]
are sufficient to guarantee $Q_S>0$ and $c_S^2>0$.

\subsection*{4.2 Tensor Sector}
Tensor perturbations obey
\begin{equation}
\ddot h_{ij}+(3H+\nu)\dot h_{ij}+c_T^2\frac{k^2}{a^2}h_{ij}=\Pi_{ij},
\end{equation}
with $c_T^2=1$ for curvature-only elastic sectors.  
This satisfies the GW170817 constraint $c_{\mathrm{GW}}=c$.

\subsection*{4.3 Vector Sector}
No new dynamical vector modes arise; standard decay persists.

\section{Hamiltonian Positivity}
Mapping to scalar–tensor representation with $\phi=f_R$ yields a Brans–Dicke-like theory with $\omega_{\mathrm{BD}}>0$ and potential $V(\phi)$ bounded below.  
This eliminates Ostrogradsky instabilities and ensures the Hamiltonian is positive definite around the background.

\section{Perturbative Appendix (Sketch)}
1. Compute $Q_S,c_S^2$ from second variation of the action.  
2. Demonstrate $c_T^2=1$ for curvature-only elastic terms.  
3. Identify $\rho_\sigma,p_\sigma$ via projection of $\sigma_{\mu\nu}$ on $u^\mu$.  
4. Verify closure of Friedmann system with $\nabla\cdot\sigma=0$.

\section{Summary of Mathematical Results}
\begin{itemize}
\item $\sigma_{\mu\nu}$ derived from $\mathcal{L}_{\mathrm{elastic}}$ is covariantly conserved.
\item In FLRW, it behaves as an effective fluid with evolving $w_\sigma(a)$.
\item Linear perturbations are ghost- and gradient-stable for $f_R>0,f_{RR}>0$.
\item Gravitational-wave speed $c_T=1$; no Lorentz violation.
\item The elastic sector is mathematically well-posed and consistent with the v9.0 empirical calibration.
\end{itemize}

\end{document}
